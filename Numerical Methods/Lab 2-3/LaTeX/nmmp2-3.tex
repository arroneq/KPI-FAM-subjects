% !TeX program = lualatex
% !TeX encoding = utf8
% !BIB program = biber
% !TeX spellcheck = uk_UA

\documentclass{mathreport}
% ------------------------------------------------------------------------------------------------------------
% Add Additional Packages
% ------------------------------------------------------------------------------------------------------------

\renewcommand{\theequation}{\arabic{subsection}.\arabic{equation}}
\renewcommand{\thesubsection}{\arabic{subsection}}

\usepackage{bbm} % for using indicator \mathbbm{1}

% reset equation numbering after each subsection
\counterwithin*{equation}{subsection}

% set page background color (dark read mode)
% \pagecolor[rgb]{0.118,0.118,0.118}
% \color[rgb]{0.8,0.8,0.8}

\begin{document}

\ReportPreamble{Лабораторна робота №2 -- №3}
\ReportName{Задача динамічного деформування твердого тіла}
\ReportSubject{Чисельні методи математичної фізики}

\AuthorInfo{Студент 5 курсу, групи КМ-31мн,}
\AuthorName{Цибульник Антон Владиславович}

\SupervisorInfo{Професор кафедри ПМА,}
\SupervisorName{Ориняк Ігор Володимирович}

% warning: in order to fit the text in the very right side of a page, set the longest label
\TheLongestLabel{Цибульник Антон Владиславович}

\import{Title/}{title}

\tableofcontents

\newpage

\section{Постановка задачі}
% \addcontentsline{toc}{section}{Постановка задачі}

У лабораторній роботі розглядається задача динамічного деформування балки в часі. Для балки, що згинається, кожна точка $s$ балки довжиною $2L=20$ у кожен момент часу $t$ характеризується чотирма параметрами: переміщенням~$W(s,t)$, кутом згинання $\theta(s,t)$, згинальним моментом $M(s,t)$ та внутрішньою поперечною силою $Q(s,t)$. Відтак, деформуванню балки відповідатиме така система диференціальних рівнянь:
\begin{align}\label{eq: initial d.e.}
    \frac{\partial W(s,t)}{\partial s} = \theta(s,t), &&  \frac{\partial\theta(s,t)}{\partial s} = M(s,t), && \frac{\partial M(s,t)}{\partial s} = Q(s,t), && \frac{\partial Q(s,t)}{\partial s} = \alpha(s,t),
\end{align}
де в рамках потавленої задачі розподілена сила $\alpha(s,t)$ має дві складові різної природи: силу інерції, направлену в протилежний до напрямку переміщення бік
\begin{equation}\label{eq: inertia force}
    p_{in}(s,t) = -\frac{\partial^2 W(s,t)}{\partial t^2},
\end{equation}
та зовнішню силу
\begin{equation}\label{eq: external force}
    p_{ex}(s,t) = p_0\, \delta(s-L/2) \cos{\eta t},
\end{equation}
де $p_0$ є константою, $\delta(s-s_0)$ є дельта-функцією Дірака, яка задає зосереджену силу в точці $s_0=L/2$, а параметр $\eta$ є частотою дії зовнішньої сили. 

Зауважимо, що сталі характеристики системи, такі як модуль пружності $E$, характеристика геометрії січення $J$ та площа перерізу балки $S$ покладені одниці. Також зазначимо, що дельта-функція Дірака $\delta(s-s_0)$ приймає нульове значення усюди, окрім околу точки $s_0$, де її значення сягає нескінченності:
\begin{align}\label{eq: delta Dirac}
    \delta(s-s_0)=
    \begin{cases*}
        0, & $s \neq s_0$ \\
        \infty, & $s=s_0$ \\
    \end{cases*},
\end{align}
та, крім того, в загальному випадку дельта-функція Дірака $\delta(s)$ володіє такими властивостями на області дії $D:$
\begin{align}
    & \int\limits_{D}\delta(s)\,ds = 1 \label{eq: delta Dirac feature 1} \\
    & \int\limits_{D}\delta(s-s_0)f(s)\,ds = f(s_0) \label{eq: delta Dirac feature 2}
\end{align}

Отже, звівши систему чотирьох рівнянь~\eqref{eq: initial d.e.} до одного рівняння четвертого ступеня,  моделювання динамічного деформування балки описуватиметься таким диференціальним співвідношенням:
\begin{equation}\label{eq: united d.e.}
    \frac{\partial^4 W(s,t)}{\partial s^4} + \frac{\partial^2 W(s,t)}{\partial t^2} = p_{ex}(s,t) = p_0\,\delta(s-L/2) \cos{\eta t}
\end{equation}
із чотирма граничними умовами
\begin{align}\label{eq: edge conditions}
    W(0,t)=0, && W(2L,t)=0, && \frac{\partial^2W(0,t)}{\partial s^2}=0, && \frac{\partial^2W(2L,t)}{\partial s^2}=0
\end{align}

Наостанок, балка, яка підлягає деформуванню, згідно з умовами задачі має особливість~--- проміжну опору в точці $L$. Тож на додачу до граничних умов~\eqref{eq: edge conditions} отримуємо обмеження виду
\begin{align}\label{eq: central condition}
    W(L,t)=0
\end{align}

У подальших викладках буде розглянуто пошук точного аналітичного розв'язку за методом початкових параметрів (МПП) й наближеного аналітичного розв'язку за методом зважених залишків (МЗЗ) для системи рівнянь~\eqref{eq: united d.e.} у випадку відсутності дії зовнішнього навантаження (однорідне рівняння деформації) та у випадку наявного зовнішнього навантаження (неоднорідне рівняння деформації).

\section{Ідея методу початкових параметрів (МПП)}

Метод початкових параметрів розглядає довільну систему як такі сутності: елементи; межі між елементами (кінці, вузли), де відбувається спряження дотичних елементів; границі всієї системи. При цьому для системи вводиться поняття потужності $N$ --- кількості параметрів, які визначають стан системи в кожній його точці $s$. Виокремлення окреслених вище сутностей системи відбувається поетапно разом із такими супутніми процедурами:

\begin{enumerate}
    \item Система дробиться на декілька окремих ділянок (елементів), і кожна така ділянка нумерується відповідним чином. Після цього визначаються вхідні та вихідні краї кожного елемента, а також вузли --- точки одночасного дотику декількох елементів. Іншими словами, відбувається організація обходу по елементах системи;
    \item Нумерація невідомих змінних (параметрів) на кожному із двох країв кожного елемента системи;
    \item Складання так званих рівнянь зв’язку для кожного елемента. Ці рівняння зв’язують параметри в кінцевій точці елемента зі значеннями в точці початку елемента. Рівняння зв'язку випливають з фізичних чи геометричних властивостей кожного елемента та системи в цілому;
    \item Складання рівнянь спряження в кожному вузлі системи;
    \item Складання рівнянь, що відповідають граничним умовам системи.
\end{enumerate}

Для системи потужності $N$, що складається з $K$ елементів, кількість невідомих параметрів системи складає $2KN$, адже для кожного елемента визначено невідомі змінні (параметри) на його початку та в його кінці. Відповідно, кількість складених рівнянь згідно з методом початкових параметрів має бути $2KN$.  

\section{Ідея методу зважених залишків (МЗЗ)}

Метод зважених залишків оперує рівнянням вигляду
\begin{equation}\label{eq: G(y) = f(x)}
    G(y) = f(x),
\end{equation}
де $G(y)$~--- деякий заданий лінійний диференціальний оператор над функцією $y(x)$, а $f(x)$ у правій частині є певним зовнішнім навантаженням (дією зовнішніх сил). Припускається, що функція $y(x)$ має форму суми $M$ так званих базових функцій $\phi_i(x)$, помножених на невідомі коефіцієнти $a_i$:
\begin{equation}\label{eq: y(x) approximation}
    y(x) = \sum\limits_{i=1}^{M} a_i\phi_i(x)
\end{equation}

Зауважимо, що перелік базових функцій задається так, щоб задовольнити нульові граничні умови задачі. Отже, оскільки вигляд~\eqref{eq: y(x) approximation}~--- лише наближення невідомої функції $y(x)$, вводиться поняття залишку диференціального оператора:
\begin{equation}\label{eq: R(x) residual}
    R(x) = \sum\limits_{i=1}^{M} a_i G\bigl( \phi_i(x) \bigr) - f(x)
\end{equation}

Мета методу полягає у мінімізації утвореного залишку $R(x)$ шляхом пошуку оптимальних значень коефіцієнтів $a_i$ через почергову процедуру <<зваження>> з кожною базовою функцією $\phi_i(x)$:
\begin{equation}\label{eq: R(x) minimizing}
    \int\limits_{\mathbb{R}} R(x)\, \phi_i(x)\, dx = 0,\ i=\overline{1,M}
\end{equation}

Таким чином, кількість рівнянь~\eqref{eq: R(x) minimizing} дорівнює кількості невідомих коефіцієнтів $a_i$, що дозволяє розв'язати утворену систему рівнянь та отримати наближене аналітичне рішення згідно з припущенням~\eqref{eq: y(x) approximation}.

\section{Рішення однорідного рівняння деформації}

\import{Chapters/}{chapter 1 (homo)}

\section{Рішення неоднорідного рівняння деформації}

\import{Chapters/}{chapter 2 (inhomo)}

\newpage
\section{Висновки}

У лабораторній роботі було розглянуто задачу динамічного деформування балки довжиною $2L=20$, яка за умовами задачі має проміжну опору в точці $L$ та дію зосередженої сили в точці $L/2$. Кожна точка~$s$ балки у кожен момент часу~$t$ характеризується чотирма параметрами: переміщенням~$W(s,t)$, кутом згинання~$\theta(s,t)$, згинальним моментом~$M(s,t)$ та внутрішньою поперечною силою~$Q(s,t)$.

Пошук аналітичного розв'язку динамічного деформування був виконаний за допомогою методу початкових параметрів (МПП), який охоплював два етапи. Перший етап полягав у пошуку розв'язку однорідного рівняння, а саме: пошуку власних частот коливання системи та пошуку відповідних власних форм коливання. У результаті було отримано п'ять власних форм, частина з яких демонструє унікальний характер коливання саме для випадку балки з проміжною опорою, а частина~--- демонструє складові за аналогією до випадку однопролітної балки. Другий етап МПП був присвячений пошуку розв'язку неоднорідної частини рівняння деформування за допомогою класичного підходу та шляхом розкладу функції переміщення за власними формами, знайденими на першому етапі МПП.

Для пошуку наближеного аналітичного розв'язку було використано метод зважених залишків (МЗЗ). Перш за все, було отримано декілька різних власних частот та власних форм, які почергово включали різну кількість базових функцій з експоненціальної сім'ї. Отримані за МЗЗ криві демонстрували відповідність власним формам, виведеним за методом МПП. По-друге, отриманий розв'язок неоднорідного рівняння також повністю відповідав переміщенню за аналітичним виглядом.

Наостанок, відзначимо, що пошук власних частот системи був також виконаний за допомогою альтернативного методу~--- шляхом побудови амплітудно-частотних характеристик та визначення резонансних значень частот на ній.

\newpage
\section{Програмна реалізація}

В ході дослідження було використано засоби мови програмування \texttt{Python} версії \texttt{3.8.10} в інтегрованому середовищі розробки \texttt{Visual Studio Code} версії \texttt{1.78.2}. Нижче наведені тексти ключових інструментальних програм.

\lstinputlisting[linerange={1-6}, caption={Підключення бібліотек та ініціалізація параметрів}]{Code/code.py}

\lstinputlisting[linerange={8-16}, caption={МПП~--- функції Крилова}]{Code/code.py}

\lstinputlisting[linerange={18-31}, caption={МПП~--- визначення власних частот}]{Code/code.py}

\lstinputlisting[linerange={33-48}, caption={МПП~--- визначення власних форм}]{Code/code.py}

\lstinputlisting[linerange={50-66}, caption={МЗЗ~--- визначення власних частот}]{Code/code.py}

\lstinputlisting[linerange={68-80}, caption={МЗЗ~--- визначення власних форм}]{Code/code.py}

\newpage
\lstinputlisting[linerange={82-108}, caption={МПП~--- розв'язок неоднорідної частини (класичний підхід)}]{Code/code.py}

\lstinputlisting[linerange={110-134}, caption={МПП~--- розв'язок неоднорідної частини (розклад по власним формам)}]{Code/code.py}

\lstinputlisting[linerange={136-173}, caption={МЗЗ~--- розв'язок неоднорідної частини}]{Code/code.py}

% \newpage
% \printbibliography[title={Перелік посилань}] % \nocite{*}
% \addcontentsline{toc}{subsection}{Перелік посилань}

\end{document}