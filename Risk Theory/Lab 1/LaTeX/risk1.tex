% !TeX program = lualatex
% !TeX encoding = utf8
% !BIB program = biber
% !TeX spellcheck = uk_UA

\documentclass{mathreport}
% ------------------------------------------------------------------------------------------------------------
% Add Additional Packages
% ------------------------------------------------------------------------------------------------------------

\renewcommand{\theequation}{\arabic{subsection}.\arabic{equation}}
\renewcommand{\thesubsection}{\arabic{subsection}}

\usepackage{bbm} % for using indicator \mathbbm{1}

% reset equation numbering after each subsection
\counterwithin*{equation}{subsection}

% set page background color (dark read mode)
% \pagecolor[rgb]{0.118,0.118,0.118}
% \color[rgb]{0.8,0.8,0.8}

\begin{document}

\ReportPreamble{Лабораторна робота №1}
\ReportName{Статична модель страхування від COVID-19}
\ReportSubject{Методи теорії надійності та ризику}

\AuthorInfo{Студент 5 курсу, групи КМ-31мн,}
\AuthorName{Цибульник Антон Владиславович}

\SupervisorInfo{Професор кафедри ПМА,}
\SupervisorName{Норкін Володимир Іванович}

% warning: in order to fit the text in the very right side of a page, set the longest label
\TheLongestLabel{Цибульник Антон Владиславович}

\import{Title/}{title}

\tableofcontents

\newpage

\section*{Постановка задачі}
\addcontentsline{toc}{section}{Постановка задачі}

Нехай деяка страхова компанія уклала з $N$ клієнтами містечка з населенням $L$ мешканців договори на захист від захворювання COVID-19 (з госпіталізацією) строком на один рік. Відомо, що щодня у місті в середньому хворіють $M$ людей, а $\alpha \in (0,1)$ --- частина з них, яка потребує госпіталізації. 

Вартість стандартного стаціонарного лікування одного пацієнта складає $K$ умовних одиниць. Завдання полягає у пошуку справедливої ціни на річну страховку від COVID-19.

\section*{Варіант завдання}
\addcontentsline{toc}{section}{Варіант завдання}

Згідно з порядковим номером у групі $n=3$, параметр загального капіталу компанії на рік $U$, найменша можлива ймовірність банкрутства $\varepsilon$, значення квантиля $q$, тестове значення ціни страховки $\pi^*$, а також перелічені у попередньому розділі величини визначатимуться так:

\begin{table}[H]\centering
    \begin{tblr}{
            % hlines={0.5pt,solid}, 
            % vlines={1pt,solid},
            colspec={X[c]|X[c]|X[c]|X[c]|X[c]|X[c]|X[c]|X[c]|X[c]},
            row{1-3}={mode=math},
        }
        
        N   & M      & L        & K      & U    & \alpha        & \varepsilon   & q              & \pi^* \\
        \hline
        10n & 10^3 n & 10^6 n^2 & 10^3 n & 10^4 & \frac{1}{n+1} & \frac{1}{n+1} & 1-\frac{1}{n+1} & K/N  \\
        \hline
        30  & 3000   & 9 \cdot 10^6 & 3000 & 10^4 & 0.25 & 0.25 & 0.75 & 100  \\

    \end{tblr}
    \caption{Значення параметрів системи}
    \label{table: initial values of parameters}
\end{table}

\section*{Хід дослідження}
\addcontentsline{toc}{section}{Хід дослідження}

\subsection{Ймовірність мешканця міста захворіти протягом року}

Оскільки $M$ --- середня кількість хворих на день, то для одного з $L$ мешканців міста ймовірність захворіти протягом одного дня складає $M/L$. Відтак через незалежність цих подій з дня до дня, ймовірність бути здоровим протягом усього року дорівнює такому добутку ймовірностей: $(1-\frac{M}{L})^{365}$. Імовірність протилежної події вказуватиме шукану ймовірність мешканця міста захворіти протягом року:
\begin{equation}\label{eq: p probability}
    p = 1-\left( 1-\frac{M}{L} \right)^{365}
\end{equation}

\subsection{Ймовірнісний розподіл витрачених компанією грошей протягом року}

Перш за все, розглядатимемо події хвороби одного з $N$ клієнтів компанії як серію незалежних експериментів. Тоді розподіл випадкової величини $\xi$ як кількості хворих клієнтів компанії протягом року буде підпорядковуватися біноміальному розподілу з параметрами $N$ та $p:$
\begin{equation}\label{eq: r.v. xi}
    \xi \sim Bin(N,p)
\end{equation}

Однак страхова компанія має зобов'язання виключно перед $\alpha$-часткою хворих осіб, які потребують госпіталізації. Отже, ввівши випадкову величину $\eta$ як кількість госпіталізованих клієнтів компанії протягом року, отримаємо біноміальний розподіл із такими параметрами:
\begin{equation}\label{eq: r.v. eta}
    \eta \sim Bin(N,\alpha p)
\end{equation}

Підставляючи значення параметрів згідно з варіантом, отримуємо величини $N=30$, $p=0.1146$ та $\alpha=0.25$. Графік функції розподілу випадкової величини $\eta$ зображений на Рис. \ref{pic: eta distribution}. 
\begin{figure}[H]\centering
    \begin{tikzpicture}
    \begin{axis}[
        height=0.5\linewidth,
        width=0.9\linewidth,
        xlabel={Кількість $x$ госпіталізованих пацієнтів},
        ylabel={Значення ймовірності $P(\eta \leqslant x)$},
        scale only axis,
        xmin=-1, xmax=31, 
        ymin=0.35, ymax=1.05,
        grid=both,
        grid style={draw=gray!30},
        minor grid style={draw=gray!10},
        minor x tick num=1,
        minor y tick num=3,
    ]
        \addplot[blue!80, mark=*, mark size=3] table {Data/eta distribution.txt};
    \end{axis}
\end{tikzpicture}
    \caption{Функція розподілу випадкової величини $\eta \sim Bin(30,0.0287)$}
    \label{pic: eta distribution}
\end{figure}

\newpage
З графіка бачимо, що, оскільки ймовірність захворіти $p$ на додачу до відсотка госпіталізованих $\alpha$ є малими, функція розподілу зростає стрімко. Іншими словами, при заданих параметрах спостерігаємо високу ймовірність того, що хворих клієнтів буде небагато.

Повертаючись до завдання підрозділу: враховуючи ціну $K$ стаціонарного лікування від COVID-19, випадкова величина $K\eta$ задаватиме кількість витрачених компанією грошей протягом року. Функція розподілу цієї величини матиме вигляд:
\begin{equation}\label{eq: K eta cdf part 1}
    F_{K\eta}\left( y \right) = P\left( K\eta \leqslant y \right) = P\left( \eta \leqslant \frac{y}{k} \right) = F_{\eta}\left( \frac{y}{K} \right) 
\end{equation}

Враховуючи вигляд функції розподілу $\eta$, матимемо:
\begin{equation}\label{eq: K eta cdf part 2}
    F_{K\eta}\left( y \right) = \sum_{k=1}^{\lfloor y/K \rfloor} C_N^k (p\alpha)^{k} (1-p\alpha)^{N-k} 
\end{equation}

\subsection{Математичне сподівання, стандартне відхилення та \linebreak $q$-квантиль кількості витрачених компанією грошей}

Шукані характеристики випадкової величини $K\eta$ задаватимуться так:
\begin{align}
    & M(K\eta) = K M\eta = KNp\alpha\\
    & \sqrt{D(K\eta)} = \sqrt{K^2 D\eta} = \sqrt{K^2Np\alpha(1-p\alpha)}\\
    & F_{K\eta}(y) = q \ \Longrightarrow \ y=F_{K\eta}^{-1}(q)
\end{align}

Враховуючи значення необхідних параметрів з Табл. \ref{table: initial values of parameters}, отримуємо: 

\begin{table}[H]\centering
    \begin{tblr}{
            % hlines={0.5pt,solid}, 
            % vlines={1pt,solid},
            colspec={c|c|c},
            row{1-2}={mode=math},
        }
        
        M(K\eta)   & \sqrt{D(K\eta)}      & 75\%\text{-ий квантиль} \\
        \hline
        2\;583 & 2\;743.47 & 1 \\

    \end{tblr}
    \caption{Числові характеристики випадкової величини $K\eta$}
    \label{table: characteristics of r.v. K eta}
\end{table}

\subsection{Середня кількість витрачених на одного клієнта грошей}

Відштовхуючись від того, що $K\eta$ задає кількість витрачених компанією грошей протягом року, тоді $M(K\eta)$ --- середня кількість витрачених компанією грошей на усіх своїх клієнтів, а відтак витрати на одного клієнта складатимуть:
\begin{equation}\label{eq: expenses per client}
    \frac{M(K\eta)}{N} = Kp\alpha = 86.1 
\end{equation}

\subsection{Ймовірність страхової компанії збанкрутіти протягом року}

Задано, що загальний капітал компанії на рік складає $U$ умовних одиниць, а ціна страховки при цьому --- $\pi^*$ умовних одиниць. Із цього випливає, що на додачу до наявного капіталу компанія отримає від $N$ клієнтів ще $N\pi^*$ умовних одиниць. Таким чином, імовірність банкрутства формулюватиметься як імовірність перевищення кількості витрачених компанією грошей протягом року $K\eta$ над сумарними збереженнями компанії $U+N\pi^*:$
\begin{equation}\label{eq: bankruptcy probability}
    P\left( K\eta > U+N\pi^* \right) = 1 - P\left( K\eta \leqslant U+N\pi^* \right) = 1 - F_{\eta}\left( \frac{U+N\pi^*}{K} \right)
\end{equation}

Підставляючи значення з Табл. \ref{table: initial values of parameters}, імовірність банкрутства дорівнюватиме:
\begin{equation}\label{eq: bankruptcy probability value}
    P\left( K\eta > U+N\pi^* \right) = 0.0015
\end{equation}

\subsection{Оптимальна ціна страховки при заданій граничній імовірності банкрутства}

Аналогічними до попереднього підрозділу міркуваннями приходимо до рівняння пошуку оптимальної ціни страховки $\widehat{\pi}$ при заданій граничній імовірності банкрутства $\varepsilon:$
\begin{equation}\label{eq: optimal policy price part 1}
    P\left( K\eta > U+N\widehat{\pi} \right) \leqslant \varepsilon
\end{equation}

Таким чином 
\begin{equation}\label{eq: optimal policy price part 2}
    1 - F_{\eta}\left( \frac{U+N\widehat{\pi}}{K} \right) \leqslant \varepsilon
\end{equation}

В результаті
\begin{equation}\label{eq: optimal policy price part 3}
    \widehat{\pi} \geqslant \frac{KF_{\eta}^{-1}(1-\varepsilon) - U}{N}
\end{equation} 

Зважаючи на те, що згідно з отриманим результатом \eqref{eq: bankruptcy probability value} імовірність банкрутства є вкрай низькою при достатньо малому (відносно ціни стаціонарного лікування) значенні ціни страховки $\pi^*=100$ умовних одиниць, можна припустити, що навіть при безплатному страхуванні усіх клієнтів компанія матиме достатньо власного капіталу $U$ для того, щоб імовірність банкрутства не перевищувала $\varepsilon=0.25$.

Тому в рамках дослідження пропонується відійти від заданого у початкових умовах відсотка госпіталізованих $\alpha=0.25$ та покласти натомість дещо інші умови у припущенні, що абсолютно кожному хворому клієнту буде надане страхування від COVID-19. 

Іншими словами, нехай $\alpha=1$, тоді найменша оптимальна ціна страховки при заданій граничній імовірності банкрутства $\varepsilon$ складатиме:
\begin{equation}\label{eq: optimal policy price value}
    \widehat{\pi} \geqslant 166.6
\end{equation} 

\subsection{Залежність імовірності банкрутства від ціни страховки}

Наостанок зобразимо на Рис. \ref{pic: bankruptcy probability distribution} залежність імовірності банкрутства \eqref{eq: bankruptcy probability} від ціни страховки:

\begin{figure}[H]\centering
    \begin{tikzpicture}
    \begin{axis}[
        height = 0.5\linewidth,
        width = 0.875\linewidth,
        xlabel={Ціна страховки},
        ylabel={Імовірність банкрутства},
        scale only axis,
        scaled y ticks=false,
        xmin=-10, xmax=510,
        ymin=-0.0005, ymax=0.0105, 
        ytick distance=0.001,
        yticklabel style={
            /pgf/number format/.cd,
            fixed,
            precision=3
        }, % set fixed precision of 2 decimal places
        grid=both,
        grid style={draw=gray!30},
        minor grid style={draw=gray!10},
        minor x tick num=1,
        minor y tick num=3,
    ]
        % \addplot[blue!80, line width=1pt] table {Data/bankruptcy probability.txt};
        \addplot[blue!80, line width=2pt] table {
            0 0.010175333947356036
            66 0.010175333947356036
        };
        \addplot[gray!50, dash pattern={on 7pt off 4pt}, line width=1pt] table {
            66 0.010175333947356036
            66 0.0015100043944361596
        };

        \addplot[blue!80, line width=2pt] table {
            67 0.0015100043944361596
            166 0.0015100043944361596
        };
        \addplot[gray!50, dash pattern={on 7pt off 4pt}, line width=1pt] table {
            166 0.0015100043944361596
            166 0.00018127121708721994
        };

        \addplot[blue!80, line width=2pt] table {
            167 0.00018127121708721994
            266 0.00018127121708721994
        };
        \addplot[gray!50, dash pattern={on 7pt off 4pt}, line width=1pt] table {
            266 0.00018127121708721994
            266 0.0
        };

        \addplot[blue!80, line width=2pt] table {
            267 0.0
            500 0.0
        };

    \end{axis}
\end{tikzpicture}
    \caption{Залежність імовірності банкрутства від ціни страховки}
    \label{pic: bankruptcy probability distribution}
\end{figure}

\section*{Програмна реалізація}
\addcontentsline{toc}{section}{Програмна реалізація}

В ході дослідження було використано засоби мови програмування \texttt{Python} версії \texttt{3.8.10} в інтегрованому середовищі розробки \texttt{Visual Studio Code} версії \texttt{1.78.2}. Нижче наведені тексти ключових інструментальних програм.

\lstinputlisting[linerange={1-3}, caption={Підключення бібліотек}]{Code/code.py}

\lstinputlisting[linerange={5-19}, caption={Ініціалізація параметрів}]{Code/code.py}

\lstinputlisting[linerange={21-29}, caption={Побудова графіку функції розподілу випадкової величини $\eta$}]{Code/code.py}

\lstinputlisting[linerange={31-38}, caption={Обчислення основних формул та імовірностей}]{Code/code.py}

\lstinputlisting[linerange={40-48}, caption={Побудова залежності імовірності банкрутства від ціни страховки}]{Code/code.py}

% \newpage
% \printbibliography[title={Перелік посилань}] % \nocite{*}
% \addcontentsline{toc}{subsection}{Перелік посилань}

\end{document}