% !TeX program = lualatex
% !TeX encoding = utf8
% !BIB program = biber
% !TeX spellcheck = uk_UA

\documentclass{mathreport}
% ------------------------------------------------------------------------------------------------------------
% Add Additional Packages
% ------------------------------------------------------------------------------------------------------------

\renewcommand{\theequation}{\arabic{subsection}.\arabic{equation}}
\renewcommand{\thesubsection}{\arabic{subsection}}

% reset equation numbering after each subsection
\counterwithin*{equation}{subsection}

% enumerate lists like sections 
\renewcommand{\labelenumii}{\arabic{enumi}.\arabic{enumii}}
\renewcommand{\labelenumiii}{\arabic{enumi}.\arabic{enumii}.\arabic{enumiii}}
\renewcommand{\labelenumiv}{\arabic{enumi}.\arabic{enumii}.\arabic{enumiii}.\arabic{enumiv}}

% set page background color (dark read mode)
% \pagecolor[rgb]{0.118,0.118,0.118}
% \color[rgb]{0.8,0.8,0.8}

\begin{document}

\ReportPreamble{Лабораторна робота №3}
\ReportName{Квантифікація ризиків}
\ReportSubject{Методи теорії надійності та ризику}

\AuthorInfo{Студент 5 курсу, групи КМ-31мн,}
\AuthorName{Цибульник Антон Владиславович}

\SupervisorInfo{Професор кафедри ПМА,}
\SupervisorName{Норкін Володимир Іванович}

% warning: in order to fit the text in the very right side of a page, set the longest label
\TheLongestLabel{Цибульник Антон Владиславович}

\import{Title/}{title}

\tableofcontents

\newpage

\section*{Постановка задачі}
\addcontentsline{toc}{section}{Постановка задачі}

У лабораторній роботі потрібно оцінити ризики, які впливають на виконання магістерської дипломної роботи з темою <<Емпіричні баєсівські методи в застосуваннях до аналізу економічних даних>>, та квантифікувати їх за двома критеріями~--- вірогідність та наслідки~--- у вигляді матриці ризиків.

\section*{Хід дослідження}
\addcontentsline{toc}{section}{Хід дослідження}

\subsection{Окреслити план виконання дипломної роботи}

План виконання магістерської дипломної роботи виглядає таким чином:
\begin{enumerate}[label=\arabic*.]
    \item Формулювання теми:
    \begin{enumerate}[label=\arabic{enumi}.\arabic{enumii}]
        \item загальне знайомство з проблемою;
        \item попередній огляд літератури і класифікація напрямків досліджень;
        \item створення попереднього короткого плану досліджень;
        \item розробка науково-технічного завдання;
        \item складання календарного плану наукового дослідження;
        \item формулювання гіпотези щодо очікуваних результатів;
        \item попередня оцінка очікуваних результатів.
    \end{enumerate}
    \item Формулювання мети і задач дослідження:
    \begin{enumerate}[label=\arabic{enumi}.\arabic{enumii}]
        \item підбір і формування бібліографічних джерел;
        \item реферування статей за відповідною тематичкою;
        \item аналіз, порівняння, критика опрацьованої інформації;
        \item узагальнення, формулювання власних суджень щодо опрацьованих питань;
        \item формулювання мети і задач дослідження.
    \end{enumerate}
    \item Математичне моделювання:
    \begin{enumerate}[label=\arabic{enumi}.\arabic{enumii}]
        \item вивчення сутності процесу чи явища, що визначає основні якості досліджуваного об'єкта;
        \item формулювання гіпотези, відбір і обґрунтування моделі;
        \item побудова математичної моделі;
        \item аналітичне дослідження моделі;
        \item теоретичний аналіз отриманих результатів.
    \end{enumerate}
    \item Аналіз і оформлення результатів наукового дослідження:
    \begin{enumerate}[label=\arabic{enumi}.\arabic{enumii}]
        \item загальний аналіз узгодженості результатів з теорією;
        \item аналіз відхилень;
        \item уточнення, розвиток попередніх гіпотез;
        \item формулювання наукових висновків;
        \item підготовка звіту чи статті.
    \end{enumerate}
\end{enumerate}

\subsection{Скласти список ризиків (небажаних подій)}

Переілік ризиків та небажаних подій можна скласти так:
\begin{enumerate}[label=R\arabic*.]
    \item Ризики виконання наукових завдань дипломної роботи.
    \begin{enumerate}[label=R\arabic{enumi}.\arabic{enumii}]
        \item Відсутність доступу до необхідних економічних даних.
        \item Труднощі у виборі та реалізації емпіричних баєсівських методів для аналізу економічних даних.
        \item Проблеми із збором емпіричних даних через обмежену доступність або невідповідність їхньої якості.
    \end{enumerate}
    \item Зміна теми дипломної роботи.
    \begin{enumerate}[label=R\arabic{enumi}.\arabic{enumii}]
        \item Невдача в розвитку ефективної методології застосування емпіричних баєсівських методів до економічних даних.
        \item Непродуктивність або невдоволеність з обраною темою дипломної роботи з боку керівника чи комісії.
    \end{enumerate}
    \item Зміна керівника дипломної роботи.
    \begin{enumerate}[label=R\arabic{enumi}.\arabic{enumii}]
        \item Конфлікти в співпраці з керівником дипломної роботи.
        \item Недостатній рівень допомоги та сприяння з боку керівника у розв'язанні проблем чи удосконаленні методології.
        \item Втрата інтересу керівника до обраної теми та методів дослідження.
    \end{enumerate}
    \item Недостатня кваліфікація студента.
    \begin{enumerate}[label=R\arabic{enumi}.\arabic{enumii}]
        \item Недостатнє знання методів і алгоритмів прикладної математики.
        \item Недостатні навички програмування.
        \item Недостане знання англійської мови.
        \item Недостане знання предметної області.
    \end{enumerate}
    \item Технічні ризики.
    \begin{enumerate}[label=R\arabic{enumi}.\arabic{enumii}]
        \item Проблеми з програмним забезпеченням для реалізації баєсівських методів.
        \item Труднощі з обробкою та аналізом великого обсягу економічних даних через обмежені ресурси або непродуктивність інструментів.
    \end{enumerate}
    \item Аналітичні ризики.
    \begin{enumerate}[label=R\arabic{enumi}.\arabic{enumii}]
        \item Труднощі у виявленні інтерпретації результатів, отриманих за допомогою емпіричних баєсівських методів.
        \item Важкість узагальнення отриманих висновків на загальноприйнятому рівні.
    \end{enumerate}
    \item Виявлення фактів недоброчесності у роботі.
    \begin{enumerate}[label=R\arabic{enumi}.\arabic{enumii}]
        \item Недостатне посилання на джерела.
        \item Випадкові співпадіння фрагментів тексту.
        \item Виявлення фактів неусвідомленого плагіату.
        \item Виявлення фактів усвідомленого плагіату.
    \end{enumerate}
    \item Форс-мажорні обставини.
    \begin{enumerate}[label=R\arabic{enumi}.\arabic{enumii}]
        \item Природні катастрофи (землетруси, повені, пожежі тощо), які можуть вплинути на доступ до літературних джерел, обладнання або навіть на місце проведення досліджень.
        \item Глобальні або регіональні кризи (економічні, політичні, епідемії), що можуть вплинути на доступність даних, можливість проведення опитувань або співпраці з іншими дослідниками.
        \item Технічні проблеми (відмова обладнання, проблеми з інтернет з'єднанням), що можуть призвести до затримок у зборі та обробці даних.
    \end{enumerate}
    \item Ризики особистого характеру.
    \begin{enumerate}[label=R\arabic{enumi}.\arabic{enumii}]
        \item Зниження мотивації студента.
        \item Проблеми зі здоров’ям студента, що обмежують час і зусилля для виконання дослідження.
    \end{enumerate}
\end{enumerate}

\subsection{Скласти матрицю ризиків}

Матриця ризиків проілюстрована на таблиці нижче (Табл.~\ref{table: risk matrix}).

\vspace{0.4cm}
\begin{table}[H]\centering
    \begin{tblr}{
            hlines={1pt,solid}, 
            vlines={1pt,solid},
            hline{1,6,8}={1-8}{2pt},
            vline{1,3,8}={1-8}{2pt},
            colspec={X[1cm,c]X[c]X[c]X[c]X[c]X[c]X[c]},
            cell{1}{1}={r=5,c=1}{c},
            cell{7}{3}={r=1,c=5}{c},
            cell{6}{1}={r=2,c=2}{c},
            rows={1cm,valign=m},
            cell{1-5}{3-7}={yellow!60},
            cell{1}{4-7}={red!60}, cell{2}{5-7}={red!60}, cell{3}{6-7}={red!60}, cell{4}{7}={red!60},
            cell{3}{3}={green!60}, cell{4}{3-4}={green!60}, cell{5}{3-5}={green!60},
        }
        
        \rotatebox{90}{Наслідки} & Дуже значні   &     R3      &        &    R1   &        &             \\
                                 & Значні        &     R2      &   R7   &    R4   &        &      R8     \\
                                 & Помірні       &             &   R5   &    R9   &        &             \\
                                 & Незначні      &             &        &         &   R6   &             \\
                                 & Дуже незначні &             &        &         &        &             \\
                                 &               & Дуже низька & Низька & Помірна & Висока & Дуже висока \\ 
                                 &               & Вірогідність \\

    \end{tblr}
    \caption{Матриця ризиків}
    \label{table: risk matrix}
\end{table}

\subsection{Визначити контрзаходи для найбільш загрозливих ризиків}

Згідно Табл.~\ref{table: risk matrix}, найбільш загрозливими виявилися ризики R1, R4, R8. Нижче наведено набір контрзаходів С1, С4, С8, який можна запропонувати задля пом'якшення відповідного ризику. Набір заходів спрямований на зменшення ймовірності небажаної події або на зменшення наслідків цієї події:
\begin{enumerate}[label=\arabic*.]
    \item[С1.] Контрзахід до ризику виконання наукових завдань дипломної роботи.
    \begin{enumerate}[label=С1.\arabic{enumii}]
        \item Створення резервного плану: pозробка альтернативного плану досліджень або методів у випадку, якщо основний план не буде успішним.
        \item Регулярні звіти та консультації: подавати регулярні звіти керівникові, щоб уникнути затримок або невдач у роботі.
    \end{enumerate}
    \item[С4.] Контрзахід недостатньої кваліфікації студента.
    \begin{enumerate}[label=С4.\arabic{enumii}]
        \item Додаткова підготовка: якщо виявиться, що студенту не вистачає певних навичок, слід виділити додатковий час для самостійного вивчення або вивчення відповідних додаткових курсів.
        \item Залучення додаткових консультантів: пошукати допомоги від інших експертів в області досліджень.
    \end{enumerate}
    \item[С8.] Контрзахід до форс-мажорних обставин.
    \begin{enumerate}[label=С8.\arabic{enumii}]
        \item Розробка кризового плану: створити план дій для непередбачених обставин, таких як природні катастрофи або інші форс-мажорні події.
        \item Резервні джерела та дані: забезпечити наявність альтернативних джерел інформації та даних, щоб зменшити вплив можливих обмежень.
        \item Резервні джерела енергії: забезпечити наявність додаткових джерел енергії на випадок блекаутів.
    \end{enumerate}
\end{enumerate}

% \newpage
% \printbibliography[title={Перелік посилань}] % \nocite{*}
% \addcontentsline{toc}{section}{Перелік посилань}

% \newpage
% \section*{Програмна реалізація}
% \addcontentsline{toc}{section}{Програмна реалізація}

% В ході дослідження було використано засоби мови програмування \texttt{Python} версії \texttt{3.8.10} в інтегрованому середовищі розробки \texttt{Visual Studio Code} версії \texttt{1.78.2}. Нижче наведені тексти ключових інструментальних програм.

% \lstinputlisting[linerange={1-3}, caption={Підключення бібліотек}]{Code/code.py}

% \lstinputlisting[linerange={5-17}, caption={Ініціалізація та візуалізація параметрів}]{Code/code.py}

% \lstinputlisting[linerange={19-42}, caption={Імплементація методу Монте-Карло}]{Code/code.py}

% \lstinputlisting[linerange={44-52}, caption={Обчислення характеристик симуляції еволюції капіталу компанії}]{Code/code.py}

\end{document}