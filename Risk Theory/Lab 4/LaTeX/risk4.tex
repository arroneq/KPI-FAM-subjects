% !TeX program = lualatex
% !TeX encoding = utf8
% !BIB program = biber
% !TeX spellcheck = uk_UA

\documentclass{mathreport}
% ------------------------------------------------------------------------------------------------------------
% Add Additional Packages
% ------------------------------------------------------------------------------------------------------------

\renewcommand{\theequation}{\arabic{subsection}.\arabic{equation}}
\renewcommand{\thesubsection}{\arabic{subsection}}

% reset equation numbering after each subsection
\counterwithin*{equation}{subsection}

% enumerate lists like sections 
\renewcommand{\labelenumii}{\arabic{enumi}.\arabic{enumii}}
\renewcommand{\labelenumiii}{\arabic{enumi}.\arabic{enumii}.\arabic{enumiii}}
\renewcommand{\labelenumiv}{\arabic{enumi}.\arabic{enumii}.\arabic{enumiii}.\arabic{enumiv}}

% set page background color (dark read mode)
% \pagecolor[rgb]{0.118,0.118,0.118}
% \color[rgb]{0.8,0.8,0.8}

\begin{document}

\ReportPreamble{Лабораторна робота №4}
\ReportName{Моделювання еволюції капіталу страхової компанії за допомогою ланцюгів Маркова}
\ReportSubject{Методи теорії надійності та ризику}

\AuthorInfo{Студент 5 курсу, групи КМ-31мн,}
\AuthorName{Цибульник Антон Владиславович}

\SupervisorInfo{Професор кафедри ПМА,}
\SupervisorName{Норкін Володимир Іванович}

% warning: in order to fit the text in the very right side of a page, set the longest label
\TheLongestLabel{Цибульник Антон Владиславович}

\import{Title/}{title}

\tableofcontents

\newpage

\section*{Теоретичні відомості про марковські моделі}
\addcontentsline{toc}{section}{Теоретичні відомості про марковські моделі}

Нехай $\left\{ U_t \right\}_{t=\overline{1,T}}$~--- послідовність випадкових величин зі значеннями в множині $E=\{ 0,1,2,\ldots,M \}$. Послідовність $\left\{ U_t \right\}_{t=\overline{1,T}}$ утворює ланцюг Маркова, якщо $\forall t\geqslant 2 \quad \forall u_1,u_2,\ldots,u_{t+1} \in E$ виконується так звана марковська властивість:
\begin{equation}\label{eq: Markov chain definition}
    P\left( U_{t+1}=u_{t+1} \,|\, U_t=u_t,\ldots,U_1=u_1 \right) = P\left( U_{t+1}=u_{t+1} \,|\, U_t=u_t \right)
\end{equation}

Множину $E$ називають множиною станів ланцюга, а випадкова величина $U_t$ трактується як стан системи в момент часу $t$. Надалі у ході дослідження по замовчуванню розглядатиметься так званий однорідний ланцюг Маркова, для якого ймовірності переходу $P\left( U_{t+1}=j \,|\, U_{t}=i \right)$ з одного стану $i \in E$ в інший стан $j \in E$ не залежать від $t$. Матриця $P$, складена із імовірностей перехіду $p_{ij} = P\left( U_{t+1}=j \,|\, U_{t}=i \right)$, називається матрицею перехідних імовірностей:
\begin{equation}\label{eq: transition matrix}
    P = \Bigl( p_{ij} \Bigr)_{i,j \in E} = \Bigl( P\left( U_{t+1}=j \,|\, U_{t}=i \right) \Bigr)_{i,j \in E}
\end{equation} 

Ця матриця є стохастичною, тобто
\begin{equation}\label{eq: stohastic conditions}
    \forall i,j \in E: p_{ij} \geqslant 0\ \text{ та }\ \forall i \in E: \sum\limits_{j \in E} p_{ij} = 1
\end{equation}

Окрім матриці $P$, ланцюг Маркова визначається вектором початкового розподілу ймовірностей:
\begin{equation}\label{eq: initial distribution}
    \lambda = \Bigl( \lambda_i \Bigr)_{i \in E} = \Bigl( P\left( U_{1} = i \right) \Bigr)_{i \in E}
\end{equation}

Скінченновимірні розподіли ланцюга Маркова повністю визначаються матрицею перехідних імовірностей $P$ та вектором початкового розподілу ймовірностей~$\lambda$, тобто $\forall t\geqslant 2 \quad \forall u_1,u_2,\ldots,u_{t} \in E:$
\begin{equation}\label{eq: finite-dimensional distributions}
    P\left( U_{1}=u_{1},U_{2}=u_{2},\ldots,U_{t-1}=u_{t-1},U_{t}=u_{t} \right) = \lambda_{u_1} \cdot p_{u_1 u_2} \cdot \ldots \cdot p_{u_{t-1}u_{t}} 
\end{equation}

Відтак, імовірність переходу зі стану $i \in E$ в момент часу $1$ до стану $j \in E$ в моменту часу $T$ виражатиметься через відповідний компонент матриці $P$, піднесеної до степені $T:$
\begin{equation}\label{eq: from t=1 to t=T}
    P\left( U_{T}=j \,|\,U_{1}=i \right) = \left( P^{T} \right)_{ij}
\end{equation}

Як наслідок, за формулою повної ймовірності безумовна імовірність перебування стану ланцюга в момент часу~$T$ матиме вигляд:
\begin{equation}\label{eq: point state probability}
    P\left( U_{T}=j \right) = \sum\limits_{i \in E} P\left( U_1=i \right) P\left( U_T=j \,|\, U_1=i \right) = \sum\limits_{i \in E} \lambda_i \left( P^{T} \right)_{ij}
\end{equation}

\section*{Постановка задачі}
\addcontentsline{toc}{section}{Постановка задачі}

Лабораторна робота стосується моделювання еволюції капіталу страхової компанії <<Арсенал страхування>> за допомогою ланцюгів Маркова. Мета роботи полягає у побудові ланцюга Маркова, стани якого відображають капітал компанії (у тисячах гривень). 

Відтак, динаміка такого ланцюга ототожнюватиме еволюцію капіталу страхової компанії. Як підсумок дослідження слід обчислити імовірність банкрутства компанії за певний відрізок часу $T$, або, іншими словами, імовірність ланцюга Маркова потрапити у стан <<0>> за вказану кількість кроків. 

\section*{Хід дослідження}
\addcontentsline{toc}{section}{Хід дослідження}

\subsection{Побудова ланцюга Маркова}

Нехай $\left\{ U_t \right\}_{t=\overline{1,T}}$~--- ланцюг Маркова, заданий рівнянням еволюції капіталу страхової компанії:
\begin{equation}\label{eq: capital evolution}
    U_{t+1} = U_{t} + (1-a) \cdot b - \xi \cdot b,
\end{equation}
де $a$~--- задана частка премій, витрачених на обслуговування договорів страхування, валове значення премії $b$ страхової компанії <<Арсенал страхування>>
\begin{equation}\label{eq: insurance premium value}
    b = 1\;667\ \text{тис. грн.},
\end{equation} 
значення капіталу $U_t$ в початковий момент часу складає
\begin{equation}\label{eq: initial (2022) capital}
    U_1 = 576\ \text{тис. грн.},
\end{equation}
а значення випадкової величини $\xi \in \Omega$ обирається щоразу навмання серед наявної історії рівнів виплат (Рис.~\ref{pic: annual payout levels}):
\begin{equation}\label{eq: annual payout levels}
    \Omega = \{ 0.20,\, 0.29,\, 0.20,\, 0.21,\, 0.26,\, 0.31,\, 0.36,\, 0.42,\, 0.43 \}
\end{equation}

Множина станів $E=\{ 0,1,2,\ldots,M \}$ ланцюга $\left\{ U_t \right\}_{t=\overline{1,T}}$ відповідатиме значенням капіталу компанії у тисячах гривень, де згідно умов варіанту при $N=13:$
\begin{equation}\label{eq: limit value of capital}
    M = 2U_1 + 100N = 2\;452\ \text{тис. грн.}
\end{equation}

Варто зауважити, що при використанні формули еволюції капіталу~\eqref{eq: capital evolution} слід зважати на такі обмеження~--- капітал компанії не може опускатися нижче нуля (стан <<0>> вважається дефолтом) чи перебувати вище значення~$M$ (надлишок вилучається з резервів, тобто, наприклад, виплачується у вигляді дивідендів). Крім того, капітал має бути виключно цілим числом.

\begin{figure}[H]\centering
    \begin{tikzpicture}
    \begin{axis}[
        height=0.5\linewidth,
        width=0.875\linewidth,
        xlabel={Позначка часового діапазону},         
        ylabel={Рівень виплат, \%},
        scale only axis,
        ymin=0.0, ymax=50,                         % set 0.0 as a "floor" of the y-axis
        xtick={                       
            2014, 2015, 2016, 2017, 2018, 2019, 2020, 2021, 2022
        },
        xticklabel style={
            /pgf/number format/.cd,
            1000 sep={},
        },
        grid=both,            
        grid style={draw=gray!30},                 
        minor grid style={draw=gray!10},           
        xtick align=center,                        % align x-axis tick marks to the center of the bars
        minor y tick num=3,
        xmajorgrids=false,                         
        xminorgrids=false,                         
    ]
        \addplot[              
            ybar,                                  % set the plot as a bar plot
            bar width=25pt,       
            fill=blue,       
            opacity=0.7,        
        ] coordinates {      
            (2014, 20.33)
            (2015, 29.12)
            (2016, 20.1)
            (2017, 20.42)
            (2018, 25.74)
            (2019, 30.68)
            (2020, 36.29)
            (2021, 41.93)
            (2022, 43.17)
        };
    \end{axis}
\end{tikzpicture}
    \caption{Рівні виплат страхової компанії <<Арсенал страхування>>}
    \label{pic: annual payout levels}
\end{figure}

Іншими словами, із використанням нотації округлення $[\ \cdot\ ]$ до найближчого цілого числа, формалізація умов та обмежень стосовно еволюції капіталу компанії виражатиметься таким чином:
\begin{equation}\label{eq: edited capital evolution}
    U_{t+1} = U_{t} + [(1-a) \cdot b] - [\xi \cdot b]
\end{equation}

Отже, згідно викладок у теоретичному розділі лабораторної роботи, ланцюг Маркова $\left\{ U_t \right\}_{t=\overline{1,T}}$ визначатиметься матрицею перехідних імовірностей $P$ та вектором початкового розподілу $\lambda$. 

Оскільки ланцюг стартуватиме із визначеного стану $U_1$~\eqref{eq: initial (2022) capital}, компоненти вектора $\lambda = \bigl( \lambda_i \bigr)_{i \in E}$ матимуть вигляд:
\begin{equation}\label{eq: set initial distribution}
    \lambda_i = 
    \begin{cases}
        1, & i = U_1 \\
        0, & i\neq U_1 \\
    \end{cases}
\end{equation}

Матриця перехідних імовірностей $P = \bigl( p_{ij} \bigr)_{i,j \in E}$ також будуватиметься згідно з певними логічними міркуваннями щодо динаміки капіталу компанії. Перш за все, імовірність виходу із дефолтного стану дорівнює нулю:  
\begin{equation}\label{eq: i=0 transition matrix}
    p_{ij} = P\left( U_{t+1}=j \,|\, U_{t}=i \right) = 
    \begin{cases}
        0, & i=0,\, j=\overline{1,M} \\
        1, & i=0,\, j=0 \\
    \end{cases}
\end{equation} 

Використовуючи нотацію індикаторної функції $\mathbbm{1}\{ \, \cdot \, \}$, яка довільній події ставить у відповідність число $0$ або $1$, розглянемо імовірності переходу зі стану, відмінного від дефолтного, у граничний стан, тобто $\forall i=\overline{1,M}$ та $j=M:$
\begin{align}\label{eq: j=M transition matrix}
    p_{ij} = P\left( U_{t+1}=j \,|\, U_{t}=i \right) & = P\left( U_{t} + [(1-a) \cdot b] - [\xi \cdot b] \geqslant M \,|\, U_{t}=i \right) = \notag \\
    & = P\left( i + [(1-a) \cdot b] - [\xi \cdot b] \geqslant M \right) = \notag \\
    & = \frac{1}{|\Omega|}\sum\limits_{k=1}^{|\Omega|}\mathbbm{1}\left\{ i + [(1-a) \cdot b] - [\xi_k \cdot b] \geqslant M \right\}
\end{align} 

Іншими словами, імовірність переходу в стан $j=M$ тим вища, чим ближче поточний стан до гарничного стану. Аналогічні міркування стосуються імовірності переходу зі стану, відмінного від дефолтного, у дефолтний стан, тобто при $\forall i=\overline{1,M}$ та $j=0:$ 
\begin{align}\label{eq: j=0 transition matrix}
    p_{ij} = P\left( U_{t+1}=j \,|\, U_{t}=i \right) & = P\left( U_{t} + [(1-a) \cdot b] - [\xi \cdot b] \leqslant 0 \,|\, U_{t}=i \right) = \notag \\
    & = P\left( i + [(1-a) \cdot b] - [\xi \cdot b] \leqslant 0 \right) = \notag \\
    & = \frac{1}{|\Omega|}\sum\limits_{k=1}^{|\Omega|}\mathbbm{1}\left\{ i + [(1-a) \cdot b] - [\xi_k \cdot b] \leqslant 0 \right\}
\end{align} 

Наостанок, для усіх інших станів $\forall i=\overline{1,M}$ та $j\neq 0,M$ імовірності переходу  обчислюватимуться таким чином:
\begin{align}\label{eq: j transition matrix}
    p_{ij} = P\left( U_{t+1}=j \,|\, U_{t}=i \right) & = P\left( U_{t} + [(1-a) \cdot b] - [\xi \cdot b] = j \,|\, U_{t}=i \right) = \notag \\
    & = P\left( i + [(1-a) \cdot b] - [\xi \cdot b] = j \right) = \notag \\
    & = \frac{1}{|\Omega|}\sum\limits_{k=1}^{|\Omega|}\mathbbm{1}\left\{ i + [(1-a) \cdot b] - [\xi_k \cdot b] = j \right\}
\end{align} 

\subsection{Модельний приклад побудови ланцюга Маркова}

Наведемо невеликий модельний приклад (поза контексту страхової компанії <<Арсенал страхування>>) для демонстрації структури утвореної матриці перехідних імовірнотстей~$P$ та вектора початкового розподілу $\lambda$. 

Припустимо, закон еволюції ланцюга Маркова задається у грошовому еквіваленті до тисяч гривень при $a = 0.1$, $b = 6$, $U_1 = 2$ та $M = 8$. Тоді множиною станів ланцюга буде $E=\{ 0,1,2,3,4,5,6,7,8 \}$, а закон еволюції матиме вид:
\begin{equation}\label{eq: test capital evolution}
    U_{t+1} = U_{t} + 5 - [6\xi]
\end{equation}

Відтак, вектор початкового розподілу дорівнюватиме
\begin{equation}\label{eq: test initial distribution}
    \lambda = (0,\, 0,\, 1,\, 0,\, 0,\, 0,\, 0,\, 0,\, 0)
\end{equation}

В свою чергу, матриця перехідних імовірностей $P$ для вказаного модельного прикладу~\eqref{eq: test capital evolution} при історії рівнів капіталу з множини~$\Omega$~\eqref{eq: annual payout levels} отримає форму:

\begin{table}[H]\centering
    \begin{tblr}{
            hlines,vlines,
            hline{1,2,11}={1pt,solid},
            vline{1,2,11}={1pt,solid},
            colspec={cccccccccc},
            rows={mode=math},
        }

            & j=0 & j=1 & j=2 & j=3  & j=4  & j=5  & j=6  & j=7  & j=8  \\
        i=0 & 1   & 0   & 0   & 0    & 0    & 0    & 0    & 0    & 0    \\
        i=1 & 0   & 0   & 0   & 0.2  & 0.45 & 0.35 & 0    & 0    & 0    \\
        i=2 & 0   & 0   & 0   & 0    & 0.2  & 0.45 & 0.35 & 0    & 0    \\
        i=3 & 0   & 0   & 0   & 0    & 0    & 0.2  & 0.45 & 0.35 & 0    \\
        i=4 & 0   & 0   & 0   & 0    & 0    & 0    & 0.2  & 0.45 & 0.35 \\
        i=5 & 0   & 0   & 0   & 0    & 0    & 0    & 0    & 0.2  & 0.75 \\
        i=6 & 0   & 0   & 0   & 0    & 0    & 0    & 0    & 0    & 1    \\
        i=7 & 0   & 0   & 0   & 0    & 0    & 0    & 0    & 0    & 1    \\
        i=8 & 0   & 0   & 0   & 0    & 0    & 0    & 0    & 0    & 1    \\

    \end{tblr}
    \label{table: transition matrix}
    \caption{Матриця перехідних імовірностей $P$ модельного прикладу~\eqref{eq: test capital evolution}}
\end{table}

\subsection{Обчислення ймовірності розорення компанії}

Повертаючись до даних по страховій компанії <<Арсенал страхування>>, то, враховуючи валове значення премії $b$~\eqref{eq: insurance premium value}, задану частку премій $a=0.1$, витрачених на обслуговування договорів страхування, значення капіталу в початковий момент часу $U_1$~\eqref{eq: initial (2022) capital}, історію рівнів виплат $\xi$ з множини $\Omega$~\eqref{eq: annual payout levels} та граничне значення капіталу $M$~\eqref{eq: limit value of capital}, отримуємо закон еволюції ланцюга Маркова
\begin{equation}\label{eq: real capital evolution}
    U_{t+1} = U_{t} + 1500 - [1667 \cdot \xi]
\end{equation}
на множині станів $E=\{ 0,1,2,\ldots,2452 \}$. При цьому, згідно із формулою~\eqref{eq: point state probability}, імовірність банкрутства компанії за, наприклад, $T=10$ років буде обчислюватися як імовірність потрапляння ланцюга Маркова в стан <<0>> за відповідну кількість кроків. Відтак, зважаючи на вигляд вектора початкового розподілу $\lambda$~\eqref{eq: set initial distribution}, шукана імовірність обчислюватиметься через побудовану матрицю перехідних імовірностей~$P:$
\begin{equation}\label{eq: real point state probability}
    P\left( U_{10}=0 \right) = \sum\limits_{i \in E} P\left( U_1=i \right) P\left( U_{10}=0 \,|\, U_1=i \right) = \left( P^{10} \right)_{U_10}
\end{equation}

У Табл.~\ref{table: bankruptcy risks} продемонстровані обчислення імовірності банкрутства при різних значеннях частки премії $a$.

\begin{table}[H]\centering
    \begin{tblr}{
            hlines,vlines,
            hline{1,2,5,8,11,14,17}={1pt,solid},
            vline{1-4}={1pt,solid},
            colspec={Q[c,4cm]Q[c,4cm]Q[c,4cm]},
            row{1}={m},
            row{2-Z}={mode=math},
            cell{2}{1}={r=3,c=1}{c},
            cell{5}{1}={r=3,c=1}{c},
            cell{8}{1}={r=3,c=1}{c},
            cell{11}{1}={r=3,c=1}{c},
            cell{14}{1}={r=3,c=1}{c},
            cell{2}{3}={r=3,c=1}{c},
            cell{5}{3}={r=3,c=1}{c},
            cell{8}{3}={r=3,c=1}{c},
            cell{14}{3}={r=3,c=1}{c},
        }

        Частка премій & Роки $T$ роботи компанії & Імовірність банкрутства \\
        a=0.1         & 10                       & 0.0                     \\
                      & 50                       &                         \\
                      & 100                      &                         \\
        a=0.3         & 10                       & 0.0                     \\
                      & 50                       &                         \\
                      & 100                      &                         \\
        a=0.5         & 10                       & 0.0                     \\
                      & 50                       &                         \\
                      & 100                      &                         \\
        a=0.7         & 10                       & 0.14                    \\
                      & 50                       & 0.45                    \\
                      & 100                      & 0.56                    \\
        a=0.9         & 10                       & 1.0                     \\
                      & 50                       &                         \\
                      & 100                      &                         \\

    \end{tblr}
    \label{table: bankruptcy risks}
    \caption{Імовірності банкрутства для компанії <<Арсенал страхування>>}
\end{table}

% \newpage
% \printbibliography[title={Перелік посилань}] % \nocite{*}
% \addcontentsline{toc}{section}{Перелік посилань}

\newpage
\section*{Програмна реалізація}
\addcontentsline{toc}{section}{Програмна реалізація}

В ході дослідження було використано засоби мови програмування \texttt{Python} версії \texttt{3.8.10} в інтегрованому середовищі розробки \texttt{Visual Studio Code} версії \texttt{1.78.2}. Нижче наведені тексти ключових інструментальних програм.

\lstinputlisting[linerange={1-2}, caption={Підключення бібліотек}]{Code/code.py}

\lstinputlisting[linerange={4-16}, caption={Ініціалізація та візуалізація параметрів}]{Code/code.py}

\lstinputlisting[linerange={18-37}, caption={Імплементація побудови матриці перехідних імовірнсотей $P$}]{Code/code.py}

\lstinputlisting[linerange={39-40}, caption={Зведення матриці до степеня $T$}]{Code/code.py}

\lstinputlisting[linerange={42-49}, caption={Обчислення імовірностей банкрутства}]{Code/code.py}

\end{document}