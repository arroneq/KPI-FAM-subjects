\subsection*{Задача 5. Gibbs Sampler for Gamma-Poisson model}
\addcontentsline{toc}{section}{Задача 5. Gibbs Sampler for Gamma-Poisson model}

\setcounter{subsection}{5}
\setcounter{equation}{0}

Нехай задано Баєсову модель такого вигляду:
\begin{align}
    & X_{ij} \,|\, \theta_i \overset{\scalebox{0.5}{ind}}{\sim} \mathrm{Poiss}(\theta_i),\ i=\overline{1,n},\ j=\overline{1,m}, \label{task 5 - eq: Bayes model Xij} \\
    & \theta_i \,|\, \sigma \overset{\scalebox{0.5}{i.i.d.}}{\sim} \mathrm{Gamma}(k,\sigma),\ i=\overline{1,n}, \label{task 5 - eq: Bayes model theta} \\
    & \sigma^{-1} \sim \mathrm{Exp}(1), \label{task 5 - eq: Bayes model sigma}
\end{align}
при цьому $k$ є відомим. Іншими словами, для кожної згенерованої випадкової величини~$\sigma$ з експоненціального розподілу генерується значення~$\theta_i$ з Гамма-розподілу, і надалі~--- набір випадкових величин~$X_{i1},\ldots,X_{im}$ з розподілу Пуасона з параметром~$\theta_i$.

\subsubsection*{Завдання (a): Gibbs Sampler}
\addcontentsline{toc}{subsection}{Завдання (a): Gibbs Sampler}

Завдання полягає у тому, щоб навести повні викладки одного кроку вибірки Гіббса для генерування апостеріорних розподілів параметрів~$\theta_i$ та~$\sigma$. Однак, перш ніж переходити до алгоритму, наведемо явний вигляд розподілів, які розглядатимуться у подальших міркуваннях.  

Дискретна випадкова величина~$\xi$ має розподіл Пуасона з параметром~$\lambda$, якщо: 
\begin{equation}\label{task 5 - eq: dPoisson}
    \xi \sim Poiss(\lambda) \ \Longleftrightarrow \ P(\xi=x) = \frac{\lambda^{x}e^{-\lambda}}{x!} 
\end{equation}

Неперервна випадкова величина~$\eta$ має Гамма-розподіл з <<shape parameter>> $k>0$ та <<scale parameter>>~$\sigma>0$, якщо:
\begin{equation}\label{task 5 - eq: dGamma}
    \eta \sim \mathrm{Gamma}(k,\sigma) \ \Longleftrightarrow \ f_{\eta}(x) = \frac{x^{k-1}}{\sigma^{k}\Gamma(k)}\, e^{-x/\sigma}\, \mathbbm{1}(x>0)
\end{equation}

Кажуть, що неперервна випадкова величина~$\eta^{-1}$ має обернений Гамма-розподіл з параметрами~$k>0$ та~$\sigma>0$, якщо:
\begin{equation}\label{task 5 - eq: dIG}
    \eta^{-1} \sim \mathrm{IG}(k,\sigma) \ \Longleftrightarrow \ f_{\eta^{-1}}(x) = \frac{x^{-k-1}}{\Gamma(k)}\,\sigma^{k} e^{-\sigma/x}\, \mathbbm{1}(x>0)
\end{equation}

Наостанок наведемо вигляд щільности експоненціального розподілу з параметром~$\lambda$ для неперервної випадкової величини $\zeta:$
\begin{equation}\label{task 5 - eq: dExp}
    \zeta \sim \mathrm{Exp}(\lambda) \ \Longleftrightarrow \ f_{\zeta}(x) = \lambda e^{-\lambda x}\, \mathbbm{1}(x>0),
\end{equation}
при цьому випадкова величина $\zeta^{-1}$ матиме щільність виду
\begin{equation}\label{task 5 - eq: dInvExp}
    \zeta^{-1} \sim \mathrm{Exp}(\lambda) \ \Longleftrightarrow \ f_{\zeta^{-1}}(x) = -\lambda x^{-2} e^{-\lambda/x}\, \mathbbm{1}(x>0)
\end{equation}

Тож нехай ініційовано пару значень~$\left( \sigma^{(0)};\theta_1^{(0)},\ldots,\theta_n^{(0)} \right)$. Тоді один крок вибірки Гіббса складатиметься із таких пунктів:
\begin{enumerate}
    \item Використовуючи значення~$\sigma^{(0)}$, згенерувати значення~$\theta_i^{(1)}$ з так званого <<full conditional distribution for parameter~$\theta_i$>>~--- $f(\theta_i \,|\, \sigma^{(0)}; X_{i1},\ldots,X_{im})$; 
    \item Маючи значення~$\theta_i^{(1)}$, згенерувати значення~$\sigma^{(1)}$ з <<full conditional distribution for parameter~$\sigma$>>~--- $f(\sigma \,|\, \theta_1^{(1)},\ldots,\theta_n^{(1)}; X_{11},\ldots,X_{1m};\ldots;X_{n1},\ldots,X_{nm})$; 
\end{enumerate}

Зауважимо, що повний умовний розподіл одного параметра пропорційний сумісному розроділу, в якому інший параметр вважається фіксованим. Продемонструємо це, застосувавши ланцюгове правило:
\begin{equation}\label{task 5 - eq: full theta conditional distribution}
    \underbracket{f(\theta_i, \sigma \,|\, X_{i1},\ldots,X_{im})}_{\text{сумісний розподіл}} = \underbracket{f(\theta_i \,|\, \sigma; X_{i1},\ldots,X_{im})}_{\text{повний умовний розподіл}} \underbracket{f(\sigma \,|\, X_{i1},\ldots,X_{im})}_{\text{не залежить від $\theta_i$}}
\end{equation}

Таким чином переконуємося, що повний умовний розподіл як фунція від~$\theta_i$ пропорційний сумісному розподілу, в якому значення~$\sigma$ покладено фіксованим. Аналогічним чином для параметра~$\sigma$:
\begin{equation}\label{task 5 - eq: full sigmaconditional distribution}
    \underbracket{f(\theta_i, \sigma \,|\, X_{i1},\ldots,X_{im})}_{\text{сумісний розподіл}} = \underbracket{f(\sigma \,|\, \theta_i; X_{i1},\ldots,X_{im})}_{\text{повний умовний розподіл}} \underbracket{f(\theta_i \,|\, X_{i1},\ldots,X_{im})}_{\text{не залежить від $\sigma$}}
\end{equation}

Отже, опишемо один крок вибірки Гіббса:
\begin{enumerate}
    \item Маючи~$\sigma^{(0)}$, згенеруємо значення~$\theta_i^{(1)}$ із відповідного умовного розподілу. Послідовно скористаємося такими викладками: щойно продемонтрованою властивістю~\eqref{task 5 - eq: full theta conditional distribution} у переході~(1), формулою Баєса у переході~(2), ланцюговим правилом у переході~(3) та незалежністю значень вибірки даних у кроці~(4):
        \begin{align*}\label{task 5 - eq: chain of rules under full conditional distribution}
            f(\theta_i \,|\, \sigma^{(0)}; X_{i1},\ldots,X_{im})
            & \overset{\scalebox{0.5}{(1)}}{\propto} f(\theta_i, \sigma^{(0)} \,|\, X_{i1},\ldots,X_{im}) \propto \\
            & \overset{\scalebox{0.5}{(2)}}{\propto} f(X_{i1},\ldots,X_{im} \,|\, \theta_i, \sigma^{(0)}) f(\theta_i, \sigma^{(0)}) = \\
            & \overset{\scalebox{0.5}{(3)}}{=} f(X_{i1},\ldots,X_{im} \,|\, \theta_i, \sigma^{(0)}) f(\theta_i \,|\, \sigma^{(0)}) f(\sigma^{(0)}) = \\
            & \overset{\scalebox{0.5}{(4)}}{=} \prod_{j=1}^{m} f(X_{ij} \,|\, \theta_i, \sigma^{(0)}) f(\theta_i \,|\, \sigma^{(0)}) f(\sigma^{(0)}) \stepcounter{equation}\tag{\theequation}
        \end{align*}

        З огляду на вигляд функцій щільностей заданої Баєсової моделі~\eqref{task 5 - eq: Bayes model Xij} --~\eqref{task 5 - eq: Bayes model sigma}, матимемо:
        \begin{align*}
            f(\theta_i \,|\, \sigma^{(0)}; X_{i1},\ldots,X_{im}) \propto \prod_{j=1}^{m} \frac{\theta_i^{X_{ij}} e^{-\theta_i}}{X_{ij}!} 
            & \times \frac{\theta_i^{k-1}}{\left[\sigma^{(0)}\right]^{k}\Gamma(k)}\, e^{-\theta_i/\sigma^{(0)}} \mathbbm{1}(\theta_i>0) \times \\
            & \times \left[\sigma^{(0)}\right]^{-2}e^{-1/\sigma^{(0)}} \mathbbm{1}(\sigma^{(0)}>0) \stepcounter{equation}\tag{\theequation}
        \end{align*}

        Відкидаючи множники, які не мають фунцкіональної залежності від~$\theta_i$, отримуємо такий вираз:
        \begin{align}
            f(\theta_i \,|\, \sigma^{(0)}; X_{i1},\ldots,X_{im}) \propto \theta_i^{m\overline{X}_i + k - 1}\, e^{-\theta_i(m + 1/\sigma^{(0)})}\, \mathbbm{1}(\theta_i>0)
        \end{align}

        А відтак, аналогічно до перетворень у формулі~\eqref{eq: theta posterior} з'ясовуємо вигляд апостеріорного розподілу параметра~$\theta_i:$
        \begin{equation}\label{task 5 - eq: posterior theta distribution}
            \theta_i \,|\, \sigma^{(0)}; X_{i1},\ldots,X_{im} \sim \mathrm{Gamma}\left( k+m\overline{X}_i,\frac{\sigma^{(0)}}{1+\sigma^{(0)}m} \right)
        \end{equation}

        Отже, використовуючи~$\sigma^{(0)}$, генеруємо значення~$\theta_i^{(1)}$ із розподілу~\eqref{task 5 - eq: posterior theta distribution}.

    \item Маючи~$\theta_i^{(1)}$, згенеруємо значення~$\sigma^{(1)}$ із відповідного повного умовного розподілу. В силу аналогічних кроків, як це показано для виведення формули~\eqref{task 5 - eq: chain of rules under full conditional distribution}, матимемо:
        \begin{multline*}
            f(\sigma \,|\, \theta_1^{(1)},\ldots,\theta_n^{(1)}; X_{11},\ldots,X_{1m};\ldots;X_{n1},\ldots,X_{nm}) \propto \\
            \propto f(X_{11},\ldots,X_{1m};\ldots;X_{n1},\ldots,X_{nm} \,|\, \theta_1^{(1)},\ldots,\theta_n^{(1)}; \sigma) f(\theta_1^{(1)},\ldots,\theta_n^{(1)} \,|\, \sigma) f(\sigma), \stepcounter{equation}\tag{\theequation}
        \end{multline*}
        а отже, зважаючи на незалежність випадкових величин $\theta_1^{(1)},\ldots,\theta_n^{(1)}$, отримаємо
        \begin{multline*}
            f(\sigma \,|\, \theta_1^{(1)},\ldots,\theta_n^{(1)}; X_{11},\ldots,X_{1m};\ldots;X_{n1},\ldots,X_{nm}) \propto \\
            \propto \prod_{i=1}^{n} f(X_{i1},\ldots,X_{im} \,|\, \theta_i^{(1)}, \sigma) \prod_{i=1}^{n} f(\theta_i^{(1)} \,|\, \sigma) f(\sigma), \stepcounter{equation}\tag{\theequation}
        \end{multline*}
        і як наслідок властивості вибірки даних:
        \begin{multline*}
            f(\sigma \,|\, \theta_1^{(1)},\ldots,\theta_n^{(1)}; X_{11},\ldots,X_{1m};\ldots;X_{n1},\ldots,X_{nm}) \propto \\
            \propto \prod_{i=1}^{n} \prod_{j=1}^{m} f(X_{ij} \,|\, \theta_i^{(1)}, \sigma) \prod_{i=1}^{n} f(\theta_i^{(1)} \,|\, \sigma) f(\sigma) \stepcounter{equation}\tag{\theequation}
        \end{multline*}

        Відтак, з огляду на вигляд функцій щільностей заданої Баєсової моделі, вираз розписуватиметься як:
        \begin{multline*}
            f(\sigma \,|\, \theta_1^{(1)},\ldots,\theta_n^{(1)}; X_{11},\ldots,X_{1m};\ldots;X_{n1},\ldots,X_{nm}) \propto \\
            \propto \prod_{i=1}^{n} \prod_{j=1}^{m} \frac{\left[ \theta_i^{(1)} \right]^{X_{ij}} e^{-\theta_i^{(1)}}}{X_{ij}!} \times \prod_{i=1}^{n} \frac{\left[ \theta_i^{(1)} \right]^{k-1}}{\sigma^{k}\Gamma(k)}\, e^{-\theta_i^{(1)}/\sigma} \mathbbm{1}(\theta_i^{(1)}>0) \times \\ 
            \times \sigma^{-2}e^{-1/\sigma} \mathbbm{1}(\sigma>0) \stepcounter{equation}\tag{\theequation}
        \end{multline*}
       
        Відкидаючи множники, які не мають фунцкіональної залежності від~$\sigma$, отримуємо такий вираз:
        \begin{multline*}
            f(\sigma \,|\, \theta_1^{(1)},\ldots,\theta_n^{(1)}; X_{11},\ldots,X_{1m};\ldots;X_{n1},\ldots,X_{nm}) \propto \\
            \propto \sigma^{-nk-2}\, e^{-(n\overline{\theta_i^{(1)}} + 1)/\sigma}\, \mathbbm{1}(\sigma>0) \stepcounter{equation}\tag{\theequation}
        \end{multline*}

        У виразі, наведеному вище, впізнаємо обернений Гамма-розподіл~\eqref{task 5 - eq: dIG}:
        \begin{equation}\label{task 5 - eq: posterior sigma distribution}
            \sigma \,|\, \theta_i^{(1)}; X_{11},\ldots,X_{nm} \sim \mathrm{IG}\left( nk+1, n\overline{\theta_i^{(1)}} + 1 \right)
        \end{equation}

        Отже, використовуючи набір~$\theta_i^{(1)}$, генеруємо значення~$\sigma^{(1)}$ із розподілу~\eqref{task 5 - eq: posterior sigma distribution}.
\end{enumerate}