\section{Огляд існуючих рішень за темою магістерської дисертації}
% \addcontentsline{toc}{section}{Задача 2. MCMC algorithms}

\subsection{Засоби аналізу макроекономічних даних}

\subsubsection{Класифікація макроекономічних показників}

Макроекономічні показники є важливим інструментом для вивчення економічного розвитку країни та визначення її ефективності у використанні ресурсів. Розглядаючи показники за видами економічної діяльності, можна зробити висновок про внесок різних галузей у формування макроекономічного образу країни. 

Система національних рахунків передбачає застосування таких макроекономічних показників, які характеризують результати економічної діяльності країни, зокрема України~\cite{Malyi2016}: випуск, валовий внутрішній продукт, валовий національний дохід, валовий національний наявний дохід та чистий внутрішній продукт.

\subsubsection*{Випуск (Output)}

Випуск відображає обсяг товарів і послуг, вироблених всіма підприємствами в усіх галузях економіки за певний період. Випуск включає валову додану вартість (GVA) --- товари й послуги, які купуються для використання, а не для перепродажу чи подальшої обробки --- та вартість проміжного споживання, що відображає вартість товарів та послуг, використаних у процесі виробництва.

\subsubsection*{Валовий внутрішній продукт (GDP)}

Валовий внутрішній продукт --- це сума всіх кінцевих товарів і послуг, вироблених в економіці за певний період часу, обчислена за вартістю на ринку. Цей показник вимірює первинні доходи лише від внутрішнього виробництва країни, враховує амортизацію або споживання основного капіталу. Валовий внутрішній продукт є центральним макроекономічним показником країни. Важливість GDP полягає в тому, що він надає комплексну інформацію про рівень економічної активності та ефективність використання ресурсів.

GDP складається з кількох компонентів, включаючи особисте споживання, інвестиції, державні витрати та чистий експорт (експорт мінус імпорт). Оцінюючи ці складові, економісти можуть визначити джерела економічного зростання та виявити можливі проблеми чи ризики.

Один із важливих аспектів GDP --- це врахування вартості виробництва, що дозволяє встановити внесок різних галузей економіки. Такий підхід дозволяє визначити, які сектори є основними виробниками, та допомагає сформувати стратегії для підтримки промисловості та сфери послуг.

GDP є не лише показником економічного виробництва, але й індикатором рівня життя. Високий GDP може свідчити про економічний розвиток, але важливо враховувати його розподіл серед населення.

\subsubsection*{Валовий національний дохід (GNI)}

Валовий національний дохід (GNI) є ключовим макроекономічним показником, який враховує економічний внесок країни не лише в її власних межах, але й за їхніми межами. Цей показник обчислюється як сума валового внутрішнього продукту (GDP) та доходів від іноземних інвестицій.

GNI враховує дохід від резидентів країни, які працюють за кордоном, а також дохід іноземних компаній, які працюють в межах країни. Це розширює перспективу оцінки економічної активності та дозволяє отримати повніше уявлення про внесок країни у світову економіку.

Одним з ключових аспектів GNI є його використання для розрахунку національного доходу на душу населення. Цей показник дозволяє визначити розподіл економічного добробуту серед громадян.

Аналіз GNI є важливим для визначення економічної стійкості та здатності країни залучати зовнішні інвестиції. Високий рівень GNI може свідчити про глобальну конкурентоспроможність, водночас його зростання може слугувати індикатором економічного розвитку.

\subsubsection*{Валовий національний наявний дохід (GNDI)}

Валовий національний наявний дохід (GNDI) являє собою інший аспект оцінки економічної активності країни, що доповнює поняття валового національного доходу (GNI). Як і GNI, GNDI враховує доходи, отримані резидентами країни як в її власних межах, так і за їхніми межами. Проте важливо зрозуміти різницю між цими двома показниками.

Основна відмінність між GNI та GNDI полягає в розрахунках амортизації, що відображає вартість зношення та старіння капіталу у виробництві. GNI містить цей параметр, тобто при визначенні GNI враховується сума витрат на відновлення та заміну виробничого капіталу.

З іншого боку, GNDI є показником <<чистого>> доходу, який відображається після врахування амортизації. Це означає, що GNDI надає більш точний образ реального доходу, який доступний для витрат чи збереження після усіх обов'язкових витрат, пов'язаних зі зношенням виробничих активів.

\subsubsection*{Чистий внутрішній продукт (NPD)}

Чистий внутрішній продукт (NPD) визначається як валовий внутрішній продукт (GDP) за винятком амортизації, що відображає вартість старіння та зношення виробничого капіталу.

Чистий внутрішній продукт є важливим інструментом для врахування екологічних аспектів економічного розвитку. На відміну від інших макроекономічних показників, NPD дозволяє уникнути перекосів, пов'язаних зі збільшенням виробництва коштом невідповідального використання ресурсів.

Аналізуючи NPD, економісти та урядові структури можуть визначити ефективність економічних стратегій та їхній вплив на сталість виробництва. Цей показник служить також індикатором сталого розвитку, визначаючи, наскільки економічне зростання спрямоване на забезпечення довгострокового добробуту, ураховуючи аспекти екології та сталості ресурсів.

\subsubsection{Інформаційні ресурси макроекономічних даних}

Інформаційні ресурси макроекономічних даних в Україні є невіддільною частиною економічного аналізу та прийняття урядових та бізнесових рішень. Ці ресурси містять різноманітні статистичні дані, звіти та інші документи, які дають повний образ поточного стану економіки країни.

Національний банк України є ключовим джерелом макроекономічних даних. Він публікує щомісячні та щоквартальні звіти, які включають інформацію про інфляцію, обсяг грошового обігу, обсяги зовнішньої торгівлі та інші важливі економічні показники. Ці дані допомагають визначити тренди у фінансовому секторі та грошовому ринку.

Державна служба статистики України є іншим ключовим джерелом макроекономічних даних. Вона надає інформацію про ВВП, рівень безробіття, виробництво промисловості, сільське господарство та інші соціально-економічні показники. Ці дані важливі для аналізу ефективності різних секторів економіки.

Міністерство економічного розвитку та торгівлі України також надає інформаційні ресурси, зокрема прогнози економічного розвитку, звіти про інвестиції та інші дані, що допомагають приймати урядові рішення щодо економічних стратегій.

Важливим джерелом є також міжнародні організації, такі як Міжнародний валютний фонд (МВФ) та Всесвітній банк, які надають аналітичні звіти та рекомендації з питань економічного розвитку.

Ці інформаційні ресурси використовуються для аналізу економічного стану країни, прогнозування трендів та визначення стратегій для сталого економічного розвитку. Забезпечення доступу до достовірної інформації є важливим чинником для формування ефективних політик та стратегій управління економікою України.

\subsubsection{Існуючі системи аналізу економічних даних}

\subsubsection*{Національні банки та статистичні служби}

Переважна більшість країн має свої національні банки та служби статистичного аналізу даних, які відповідають за збір, обробку та публікацію макроекономічних даних. Наприклад, в Україні це Національний банк України та Державна служба статистики України.

\subsubsection*{Міжнародні організації}

Організації, такі як Міжнародний валютний фонд (МВФ), Всесвітній банк чи Організація економічного співробітництва та розвитку (ОЕСР), також надають аналітичні звіти та статистичні дані з економічного розвитку країн по всьому світу.

\subsubsection*{Електронні ресурси}

У мережі Інтернет доступні численні електронні ресурси для аналізу та візуалізації економічних даних. До прикладу, це можуть бути вебсайти урядових установ, такі як:
\begin{enumerate}
    \item Вебсайт Державної служби статистики України~\cite{ukrstat}, підрозділ <<Інфографіка>> розділу <<Доступно про статистику>>;
    \item Вебсайт Національного банку України~\cite{bankgov}, розділ <<Фінансові ринки>>;
    \item Вебсайт Міністерства фінансів України~\cite{mofgov}, підрозділ <<Макроекономічний огляд та прогноз>> розділу <<Дані та статистика>>.
\end{enumerate}

Іншим прикладом можуть слугувати вебсайти наукових організацій та аналітичних агентств:
\begin{enumerate}
    \item Інститут економіки та прогнозування НАН України~\cite{ieforg};
    \item Інститут демографії та соціальних досліджень ім. М. В. Птухи НАН України~\cite{idssorg};
    \item Інститут регіональних досліджень ім. М. І. Долішнього НАН України~\cite{irdgov}.
\end{enumerate}

\subsubsection*{Спеціалізовані програмні продукти}

Нижче наведено перелік програмних продуктів та інструментів для аналізу економічних даних, зокрема статистичні пакети та бізнес-аналітичні програми:
\begin{enumerate}
    \item SAS~\cite{sascom} --- це провідний програмний продукт для аналізу даних, який використовується в корпоративному, урядовому та науковому секторах. Він пропонує широкий спектр інструментів для аналізу статистичних даних, прогнозування, моделювання та візуалізації;
    \item IBM SPSS Statistics~\cite{spssstatistics} --- ще один популярний програмний продукт для аналізу даних;
    \item Stata~\cite{stata} --- це потужний програмний продукт для аналізу даних, який часто використовується в академічних дослідженнях.
\end{enumerate}

\subsection{Математичні інструменти аналізу макроекономічних даних  }

\subsubsection{Методи регресійного аналізу та часових рядів}

Перш за все, наведемо короткий перелік математичних інструментів аналізу макроекономічних даних за допомогою регресійного аналізу та часових рядів~\cite{Biduk2016}.

Регресійний аналіз --- це статистичний метод, який використовується для дослідження взаємозв'язку між двома або більше змінними. Він дозволяє прогнозувати значення однієї змінної на основі значення іншої або інших змінних.
Регресійний аналіз широко використовується в економічних дослідженнях для аналізу макроекономічних даних. Він може використовуватися для таких цілей:
\begin{enumerate}
    \item Аналіз трендів дозволяє визначити загальний напрямок та зміну розвитку економічного показника;
    \item Прогнозування: наприклад, регресійний аналіз може використовуватися для прогнозування темпів зростання ВВП, рівня інфляції або рівня безробіття;
    \item Результати регресійного аналізу можуть використовуватися для розробки таких заходів економічної політики, як фіскальна політика, монетарна політика або зовнішня політика.
    \item Авторегресійні моделі (AR) --- це один із видів регресійних моделей, які використовуються для аналізу часових рядів. AR-моделі визначають значення економічного показника на основі його власних попередніх значень.
    \item Модель ARMA (Autoregressive Moving Average) --- це модель, яка поєднує в собі autoregressive (AR) і moving average (MA) компоненти. ARMA-моделі можуть бути використані для аналізу часових рядів, які мають як трендову складову, так і випадкову (шумову) складову.
    \item Модель ARIMA (Autoregressive Integrated Moving Average) --- це модель, яка поєднує в собі autoregressive (AR), moving average (MA) і integrated (I) компоненти. I-компонент відповідає за інтегрування часового ряду, тобто за видалення трендової складової.
    \item ARIMA-моделі можуть бути використані для аналізу часових рядів, які мають істотну трендову складову. Наприклад, ARIMA-модель може бути використана для прогнозування темпів зростання ВВП на основі темпів зростання ВВП у попередні періоди з урахуванням трендової складової.
\end{enumerate}

\subsubsection{Баєсівський та частотний статистичні підходи до обробки даних}

Серед існуючих теоретичних підходів, які використовуються для побудови оцінок показників різних математичних моделей на основі статистичних даних відрізняють так звані частотний та баєсівський підходи. 

Частотний підхід у статистичних методах обробки даних трактує невідомий параметр як невідому, але фіксовану величну. Акцент при цьому ставиться на повторюваності подій та довгострокових частотах випадкових явищ, які описують невідомий параметр.

Цей підхід виступає в опозиції до баєсівського підходу, ідея якого полягає у тому, щоб оцінити деякий невідомий параметр певної моделі (наприклад, економічної моделі), вважаючи цей параметр випадковою величиною. 

Спираючись на апріорний розподіл параметра (певні початкові уявлення про випадковий характер невідомої величини), а також на наявні статистичні дані, беєсівські методи мають на меті віднайти апостеріорний розподіл параметра, тобто розподіл параметра за умови спостереження наявних даних. У Табл.~\ref{table: frequency vs Bayes approaches} наведено порівняння частотного та баєсівського підходів.

% \vspace{0.4cm}
% \begin{table}[H]\centering
    \begin{longtblr}[
            caption = {Порівняння частотного та баєсівського підходів}, 
            label = {table: frequency vs Bayes approaches},
        ]{
            hlines={1pt,solid}, 
            vlines={1pt,solid},
            % hline{4-6}={1-5}{0pt},
            colspec={X[c]X[c]X[c]},
            % cell{1}{1}={r=2,c=1}{c},
            % cell{1}{2}={r=1,c=2}{c},
            % cell{1}{4}={r=1,c=2}{c},
            row{1}={m},
            % row{2-3}={mode=math},
        }

                             & Частотний підхід   & Баєсівський підхід \\
        Тлумачення параметра 
            & Фіксована величина
            & Випадкова величина \\
        Інформація на вході  
            & Статистичні дані
            & Статистичні дані та експертні знання (виражених в апріорному розподілі) \\
        Тлумачення результатів  
            & На основі заданої вибірки формуються висновки про значення параметрів генеральної сукупності даних
            & Результати стосуються оцінок параметрів конкретного набору даних \\ 
        Побудова оцінок параметрів  
            & Точкові статистичні оцінки з довірчими інтервалами
            & Апостеріорний розподіл, що відображає невизначеність та оновлені знання про параметр \\ 
        Розмір вибірки даних  
            & Чутливий до розміру вибірки
            & Менш чутливий, доцільний з невеликими вибірками \\ 

    \end{longtblr}
    % \caption{Порівняння частотного та баєсівського підходів}
    % \label{table: frequency vs Bayes approaches}
% \end{table}

Таким чином, особливо у складних моделях, частотний підхід є менш гнучким порівняно з баєсівський підходом, який дозволяє враховувати додаткові експертні знання та оновлювати ймовірності на основі нових даних.

На практиці, при використанні баєсівських методів виникають два питання: яким чином слід обирати апріорний розподіл та що робити у випадку, якщо апостеріорний розподіл має складний для аналізу аналітичний вигляд. 

Друге питання вирішується завдяки алгоритмам побудови марковських ланцюгів методом Монте-Карло. Наступні ж підрозділи стосуватимуться огляду існуючих практичних рішень~\cite{Robert2011} при виборі апріорного розподілу у баєсівському підході.

\subsubsection*{Informative Priors}

Розподіли, які включають конкретні апріорні знання та переконання про параметри моделі, називають інформативними апріорними розподілами (Informative Priors). Іншими словами, інформативні апріорні розподіли містять інформацію про те, які значення параметрів є більш або менш ймовірними, виходячи з попереднього знання або досвіду. 

Інформативні розподіли можуть допомогти покращити точність і надійність результатів Баєсівського аналізу. Однак, якщо апріорний розподіл є занадто інформативним, він може призвести до суб'єктивності результатів аналізу. Отже, такі розподіли варто використовувати лише при глибокому розумінні проблеми та у ситуаціях, де експертне знання не є обмеженим.

\subsubsection*{Conjugate Priors}

Апріорний розподіл є спряженим (Conjugate Prior) для певної сім'ї розподілів, якщо апріорний та апостеріорний розподіли належать до тієї ж самої сім’ї розподілів.

Спряжені розподіли корисні тим, що ми завжди отримуємо апостеріорні розподіли в аналітичній формі. В цьому і полягає математична зручність їхнього використання.

До прикладу, наведемо деякі поширені пари спряжених розподілів: спряжений до експоненціальної сім'ї буде також експоненціальний розподіл; нормальний розподіл матиме нормальний розподіл в якості спряженого;  спряженим до біноміального розподілу буде Бета-розподіл, спряженим до розподілу Пуассона --- Гамма-розподіл тощо.

Однак, на практиці існують численні випадки, коли модель не можна формалізувати певною парою загальновідомих розподілів.

\subsubsection*{Non-informative Priors}

Неінформативні / невпливові апріорні розподіли (Non-informative Priors) є концепцією апріорних розподілів, які передбачають мінімальний виплив переконань дослідника, щоб надати простоту та об'єктивність у баєсівському аналізі.

Іншими словами, основна ідея застосування цих розподілів полягає у тому, щоб обирати апріорний розподіл, який не має значущого впливу на аналіз результатів у випадку, наприклад. обмежених експертних знань про параметри моделі. Це особливо важливо у ситуаціях, коли об’єктивність є пріоритетом, і дослідник хоче, щоб апріорний розподіл не вніс суттєвого впливу на кінцеві результати. Нижче наведено деякі приклади неінформативних розподілів, таких як Diffuse Priors та Jeffreys’ Prior.

\subsubsection*{Diffuse Priors}

Пласкі апріорні розподіли (Diffuse Priors) використовуються тоді, коли дослідник має дуже обмежене або жодне попереднє знання про параметри моделі та прагне досягти максимальної об'єктивності у вивченні даних. Пласкі апріорні розподіли мають великий розкид, тобто рівномірну щільність по всьому простору значень  параметра. Прикладом може слугувати нормальний розподіл із нульовим математичним сподіванням та, умовно, нескінченною дисперсією. Такі розподіли дозволяють наявним статистичним даним безпосередньо впливати на вигляд апостеріорного розподілу.

\subsubsection*{Jeffreys’ Prior}

Цей невпливовий апріорний розподіл  відображає нейтральність щодо параметрів моделі за рахунок властивості інваріантності відносно параметризації. Апріорний розподіл Гарольда Джеффрісона (Jeffreys’ Prior) слугує загальним правилом для вибору Non-informative: розподіл обирається пропорційно квадратному кореню з визначника інформаційної матриці Фішера.

\subsubsection*{Empirical Prior}

Водночас емпіричні баєсівські методи замість того, щоб жорстко обирати певний апріорний розподіл, полягають у налаштуванні форми апріорного розподілу на основі вхідних даних.

Іншими словами, ідея полягає в тому, щоб уточнити гіперпараметри розподілу певного показника шляхом аналізу наявних даних, тобто, фактично, використати дані двічі --- перший раз для апріорної оцінки параметрів, а другий раз --- для оновлення й виведення апостеріорного розподілу. 

Такий гібридний підхід дозволяє отримати у певних випадках більш оптимальні результати. Таким чином, наступні підрозділи охоплюватимуть специфіку, методи та підходи саме емпіричних баєсівських методів.

\subsection*{Висновки до розділу}
\addcontentsline{toc}{subsection}{Висновки до розділу}

Серед існуючих математичних інструментів аналізу макроекономічних даних було виділено методи на основі регресійного аналізу та часових рядів, а також статистичні методи баєсівського підходу. Своєю чергою, було продемонстровано, що баєсівський підхід має численні альтернативи для математичної інтерпретації апріорних знань дослідника. 

Для розв'язання поставленої в рамках магістерської дисертації задачі, в якості найбільш ефективного методу було обрано групу баєсівських методів, а вже серед них --- емпіричні баєсівські методи.